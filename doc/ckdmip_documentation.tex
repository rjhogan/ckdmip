\documentclass[twoside]{article}
\usepackage[colorlinks=true,linkcolor=blue,citecolor=blue]{hyperref}
\usepackage{natbib}
\usepackage{times}
\usepackage{listings}
\usepackage[table]{xcolor}
\usepackage{color}
\usepackage{marginnote}
\usepackage{rotating}
\usepackage{lipsum}
\usepackage{longtable}
\usepackage{array}
\newcolumntype{L}{>{\raggedright\arraybackslash\hangindent=1em}}
\newcolumntype{M}{>{\raggedright\arraybackslash\hangindent=1em\ttfamily}}
\newcolumntype{X}{>{\nullfont}c}
\def\tablesetup{\rowcolors{2}{light-gray}{light-gray}\footnotesize}
% Use proper underscore character
\chardef\_=`\_
% Set math in Times Roman
\DeclareSymbolFont{letters}{OML}{ptmcm}{m}{it}
\DeclareSymbolFont{operators}{OT1}{ptmcm}{m}{n}
%\DeclareSymbolFont{bold}     {OML}{ptmcm}{b}{it}
\DeclareMathAlphabet{\mathbf}{OT1}{ptm}{b}{n}
% Page set up
\setlength{\oddsidemargin}{0cm} %{0.5cm}
\setlength{\evensidemargin}{0cm} %{0.5cm}
\setlength{\topmargin}{-2cm}
\setlength{\textheight}{24cm}
\setlength{\textwidth}{16cm}
\setlength{\marginparsep}{0.5cm}
\setlength{\marginparwidth}{0cm}
\setlength{\parindent}{1em}
\setlength{\parskip}{0cm}
\renewcommand{\baselinestretch}{1.1}
\sloppy

% Configure appearance of code listings
\definecolor{light-gray}{gray}{0.92}
\def\codesize{\small}
\def\codetabsize{\footnotesize}
\lstset{language=Fortran,
  backgroundcolor=\color{light-gray},
  basicstyle=\footnotesize\ttfamily,
  numbersep=5pt,
  xleftmargin=0cm,
  xrightmargin=0cm,
  emph={true,false,include,to},
  emphstyle=\relax}
\lstset{showstringspaces=false}

% Page headers
\usepackage{fancyhdr}
\pagestyle{fancy}
\renewcommand{\headrulewidth}{0.5pt}
\renewcommand{\sectionmark}[1]{\markright{\thesection.\ #1}}
\renewcommand{\subsectionmark}[1]{}
\fancyhead[RO,RE]{\thepage}
\fancyfoot[C]{}

% Symbols and macros
\def\code#1{{\codesize\texttt{#1}}}
\def\codetab#1{{\codetabsize\texttt{#1}}}
\def\codeemph#1{{\codesize\texttt{\textbf{#1}}}}
\def\codetabemph#1{{\codetabsize\texttt{\textbf{#1}}}}
\def\textemph#1{\textbf{#1}}
\def\citem#1{\item[{\codesize\texttt{#1}}]}
\def\codestyle#1{\texttt{#1}}
\renewcommand\thefootnote{\relax}
\reversemarginpar

% Title material
\title{CKDMIP: Technical Guide}

\author{Robin J. Hogan\\ \emph{European Centre for Medium Range
    Weather Forecasts, Reading, UK}}

\date{\normalsize Document version 1.0 (1 June 2021) applicable to CKDMIP
  software version 1.0\thanks{This document is copyright
    \copyright\ European Centre for Medium Range Weather Forecasts
    2019-2021. If you have any queries about CKDMIP that are not answered
    by this document, or by \cite{Hogan+2020}, or by the information
    on the CKDMIP home page, then please email me at
    \href{mailto:r.j.hogan@ecmwf.int}{\texttt{r.j.hogan@ecmwf.int}}.}\\
  CKDMIP home page: \url{https://confluence.ecmwf.int/display/CKDMIP}}
\begin{document}
\maketitle

%\tableofcontents
\def\thefootnote{\fnsymbol{footnote}}
\section{Introduction}
The motivation and overall structure of the Correlated K-Distribution
Model Intercomparison Project (CKDMIP) is described in detail by
\cite{Hogan+2020}.  This document describes the CKDMIP datasets
(section \ref{sec:datasets}), software (section \ref{sec:software})
and what files should be provided by project participants (section
\ref{sec:requirements}).

\section{Datasets}
\label{sec:datasets}
The CKDMIP datasets are available at \url{ftp://dissemination.ecmwf.int},
username \code{ckdmip}, password available on request from Robin Hogan
(\href{mailto:r.j.hogan@ecmwf.int}{\texttt{r.j.hogan@ecmwf.int}}).
Some of the smaller files are also available for direct download from
the CKDMIP home page.

All files are in NetCDF format.  The smaller files use the NetCDF3
`classic' format (suffix \code{nc}), which is compatible with a wider
range of software.  The larger files use NetCDF4 format, which is
actually HDF5 on disk (suffix \code{h5}), but can be accessed using
the same NetCDF function calls provided that your software is compiled
against version 4 of the NetCDF library. The latter format is required
because of its support for very large files and per-variable
compression.

Table 3 of \cite{Hogan+2020} describes the four dataset groups
provided in CKDMIP.  The first two, `Evaluation-1' and `Evaluation-2'
each contain 50 realistic atmospheric profiles, with the first
intended to be used for training and evaluation of CKD models, and the
second for independent evaluation. At the time of writing,
`Evaluation-2' has not been released. A further two, `MMM' and
`Idealized', provide additional profiles that may be useful to some
participants for generating CKD models.

\subsection{Gas concentrations}
\label{sec:conc}
The following files in the directory \code{concentrations} on the FTP
site:
\begin{verbatim}
ckdmip_evaluation1_concentrations.nc
ckdmip_mmm_concentrations.nc
ckdmip_idealized_concentrations.nc
\end{verbatim}
contain the mole fractions (in units of mol~mol$^{-1}$, equivalent to
volume mixing ratio in m$^3$~m$^{-3}$) for the nine gases considered
in CKDMIP: H$_2$O, O$_3$, N$_2$, O$_2$, CO$_2$, CH$_4$, N$_2$O, CFC11
and CFC12.  They are provided on both on `full levels'
(i.e.\ layer-averaged values) and `half levels' (the interfaces
between layers).  If a gas is `well mixed', either having a constant
mole fraction with height or varying as a prescribed function of
pressure, then the mole fraction variables also have a
\code{reference\_surface\_value} attribute containing the nominal
surface value. This then enables the software described in section
\ref{sec:software} to scale the spectral optical depths to correspond
to a different concentration. The files also contain the temperature
and pressure on full and half levels.

Note that half-level quantities are the primary variables: these are
the ones used by the Line-By-Line Radiative Transfer Model
\cite[LBLRTM;][]{Clough+2005}, and the full-level quantities were
derived from the half-level quantities: for pressure by taking the
average of the two enclosing half-level values, and for temperature
and mole fraction by performing a pressure-weighted average assuming a
linear variation with pressure within the layer.

As described by \cite{Hogan+2020}, the `Evaluation-1' dataset contains
50 realistic profiles (the \code{column} dimension is 50) of
temperature, H$_2$O and O$_3$ extracted from the ECMWF model, along
with present-day (i.e.\ the year 2020) profiles for the other gases
that are either constant with pressure (for N$_2$ and O$_2$) or vary
as a prescribed function of pressure.

The \code{column} dimension of the `MMM' dataset is 3, with these
profiles containing the median, minimum and maximum temperatures from
the original ECMWF dataset. For seven of the nine gases, a single
concentration profile is provided for each column (on both full and
half levels), which is in fact constant with column and corresponds to
present-day conditions.  But for H$_2$O and O$_3$, six variables
are provided for each gas: for H$_2$O we have
\code{h2o\_median\_mole\_fraction\_hl},
\code{h2o\_minimum\_mole\_fraction\_hl} and
\code{h2o\_maximum\_mole\_fraction\_hl}, corresponding to the
median, minimum and maximum concentrations of the original ECMWF
dataset, on half levels. The `maximum' concentrations have been capped
at liquid water saturation.  Each has values for the three columns
(corresponding to varying temperature).  The corresponding three
full-level variables have the suffix \code{fl}, and similarly for
O$_3$.

The `Idealized' dataset contains 6 columns corresponding to
temperatures 20-K apart. All of the mole-fraction variables are
constant with pressure and temperature. Since the molar absorption of
water vapour is dependent on temperature, for this gas alone 12
separate concentration profiles are provided (named
\code{h2o\_a\_mole\_fraction\_hl} to
\code{h2o\_l\_mole\_fraction\_hl}).

To assist CKDMIP participants implementing the 34 well-mixed
greenhouse-gas scenarios given in Table 2 of \cite{Hogan+2020}, the
\code{concentrations} directory on the FTP site also contains files
with the gas concentrations for each individual scenario:
\begin{verbatim}
ckdmip_evaluation1_concentrations_glacialmax.nc
ckdmip_evaluation1_concentrations_preindustrial.nc
ckdmip_evaluation1_concentrations_present.nc
ckdmip_evaluation1_concentrations_future.nc
ckdmip_evaluation1_concentrations_co2-180.nc
ckdmip_evaluation1_concentrations_co2-280.nc
ckdmip_evaluation1_concentrations_co2-560.nc
ckdmip_evaluation1_concentrations_co2-1120.nc
ckdmip_evaluation1_concentrations_co2-2240.nc
ckdmip_evaluation1_concentrations_ch4-350.nc
ckdmip_evaluation1_concentrations_ch4-700.nc
ckdmip_evaluation1_concentrations_ch4-1200.nc
ckdmip_evaluation1_concentrations_ch4-2600.nc
ckdmip_evaluation1_concentrations_ch4-3500.nc
ckdmip_evaluation1_concentrations_n2o-190.nc
ckdmip_evaluation1_concentrations_n2o-270.nc
ckdmip_evaluation1_concentrations_n2o-405.nc
ckdmip_evaluation1_concentrations_n2o-540.nc
ckdmip_evaluation1_concentrations_cfc11-0.nc
ckdmip_evaluation1_concentrations_cfc11-2000.nc
ckdmip_evaluation1_concentrations_cfc12-0.nc
ckdmip_evaluation1_concentrations_cfc12-550.nc
ckdmip_evaluation1_concentrations_co2-180-ch4-350.nc
ckdmip_evaluation1_concentrations_co2-2240-ch4-350.nc
ckdmip_evaluation1_concentrations_co2-180-ch4-3500.nc
ckdmip_evaluation1_concentrations_co2-2240-ch4-3500.nc
ckdmip_evaluation1_concentrations_co2-180-n2o-190.nc
ckdmip_evaluation1_concentrations_co2-2240-n2o-190.nc
ckdmip_evaluation1_concentrations_co2-180-n2o-540.nc
ckdmip_evaluation1_concentrations_co2-2240-n2o-540.nc
ckdmip_evaluation1_concentrations_ch4-350-n2o-190.nc
ckdmip_evaluation1_concentrations_ch4-3500-n2o-190.nc
ckdmip_evaluation1_concentrations_ch4-350-n2o-540.nc
ckdmip_evaluation1_concentrations_ch4-3500-n2o-540.nc
\end{verbatim}
These files differ from \code{ckdmip\_evaluation1\_concentrations.nc}
only in the concentration of the five well-mixed greenhouse gases (and
in fact \code{ckdmip\_evaluation1\_concentrations\_present.nc} is
identical to \code{ckdmip\_evaluation1\_concentrations.nc}).

\subsection{Absorption spectra}
\label{sec:spectra}
For each mole-fraction variable in the concentration files described
in section \ref{sec:conc}, LBLRTM has been used to compute a profile
of the spectral optical depth for that one gas.  To keep the size of
individual files less than around 10 GB, the 50 profiles of the
`Evaluation-1' dataset (in the \code{lw\_spectra/evaluation1} and
\code{sw\_spectra/evaluation1} directories on the FTP site) have been
split into five groups of 10, with the first group having file names
as follows:
\begin{verbatim}
ckdmip_evaluation1_lw_spectra_h2o_present_1-10.h5
ckdmip_evaluation1_lw_spectra_o3_present_1-10.h5
ckdmip_evaluation1_lw_spectra_n2_constant_1-10.h5
ckdmip_evaluation1_lw_spectra_o2_constant_1-10.h5
ckdmip_evaluation1_lw_spectra_co2_present_1-10.h5
ckdmip_evaluation1_lw_spectra_ch4_present_1-10.h5
ckdmip_evaluation1_lw_spectra_n2o_present_1-10.h5
ckdmip_evaluation1_lw_spectra_cfc11_present-equivalent_1-10.h5
ckdmip_evaluation1_lw_spectra_cfc12_present_1-10.h5
\end{verbatim}
and similarly in the shortwave.  The 45 longwave files amount to
around 228 GB of data in total, and the shortwave to around 112 GB.
NetCDF4/HDF5 compression has been used, with the result that the gases
with the most spectral features have the largest file sizes. In the
file names, \code{constant} indicates a profile whose mole fraction is
constant with pressure, \code{present} indicates a present-day
profile, and \code{present-equivalent} indicates that the
concentration of CFC11 has been increased to approximately represent
the radiative forcing of 38 more minor greenhouse gases
\cite[]{Hogan+2020}.

Each file contains the pressure, temperature and mole fraction on full
and half levels, for the 10 profiles, which have basically been copied
from the corresponding concentration file described in section
\ref{sec:conc}.  In the longwave files, the \code{wavenumber} variable
defines 7211999 wavenumbers in units of cm$^{-1}$, with spacings of
0.0002, 0.001 and 0.005~cm$^{-1}$ in the spectral ranges
0--1300~cm$^{-1}$, 1300--1700~cm$^{-1}$ and 1700-3260~cm$^{-1}$.  In
the shortwave files there are 3126494 wavenumbers with a spacing of
0.002, 0.001, 0.002, 0.02 and 1~cm$^{-1}$ in the spectral ranges
250--2200, 2200-2400, 2400-5150, 5150--16000 and
16000-50000~cm$^{-1}$.

The \code{optical\_depth} variable then provides the optical depth of
each atmospheric layer at each wavenumber.  Section \ref{sec:convert}
describes a tool for converting to mass extinction coefficient or
molar extinction coefficient.

The directories \code{lw\_spectra/mmm} and \code{sw\_spectra/mm}
contain the absorption spectra for the `MMM' dataset, with separate
files for the three different concentration scenarios of H$_2$O and
O$_3$:
\begin{verbatim}
ckdmip_mmm_lw_spectra_h2o_median.h5
ckdmip_mmm_lw_spectra_h2o_minimum.h5
ckdmip_mmm_lw_spectra_h2o_maximum.h5
ckdmip_mmm_lw_spectra_o3_median.h5
ckdmip_mmm_lw_spectra_o3_minimum.h5
ckdmip_mmm_lw_spectra_o3_maximum.h5
ckdmip_mmm_lw_spectra_n2_constant.h5
ckdmip_mmm_lw_spectra_o2_constant.h5
ckdmip_mmm_lw_spectra_co2_present.h5
ckdmip_mmm_lw_spectra_ch4_present.h5
ckdmip_mmm_lw_spectra_n2o_present.h5
ckdmip_mmm_lw_spectra_cfc11_present-equivalent.h5
ckdmip_mmm_lw_spectra_cfc12_present.h5
\end{verbatim}
and similarly in the shortwave. The total volume is around 25~GB in
the longwave and 12~GB in the shortwave.

The directories \code{lw\_spectra/idealized} and
\code{sw\_spectra/idealized} contain the absorption spectra for the
`Idealized' dataset, with separate files for the 12 different
concentrations of H$_2$O:
\begin{verbatim}
ckdmip_idealized_lw_spectra_h2o_constant-a.h5
ckdmip_idealized_lw_spectra_h2o_constant-b.h5
ckdmip_idealized_lw_spectra_h2o_constant-c.h5
ckdmip_idealized_lw_spectra_h2o_constant-d.h5
ckdmip_idealized_lw_spectra_h2o_constant-e.h5
ckdmip_idealized_lw_spectra_h2o_constant-f.h5
ckdmip_idealized_lw_spectra_h2o_constant-g.h5
ckdmip_idealized_lw_spectra_h2o_constant-h.h5
ckdmip_idealized_lw_spectra_h2o_constant-i.h5
ckdmip_idealized_lw_spectra_h2o_constant-j.h5
ckdmip_idealized_lw_spectra_h2o_constant-k.h5
ckdmip_idealized_lw_spectra_h2o_constant-l.h5
ckdmip_idealized_lw_spectra_o3_constant.h5
ckdmip_idealized_lw_spectra_n2_constant.h5
ckdmip_idealized_lw_spectra_o2_constant.h5
ckdmip_idealized_lw_spectra_co2_constant.h5
ckdmip_idealized_lw_spectra_ch4_constant.h5
ckdmip_idealized_lw_spectra_n2o_constant.h5
ckdmip_idealized_lw_spectra_cfc11_constant-equivalent.h5
ckdmip_idealized_lw_spectra_cfc12_constant.h5
\end{verbatim}
and similarly in the shortwave. The total volume is around 75~GB in
the longwave and 43~GB in the longwave.

\subsection{Reference broadband fluxes}
\label{sec:fluxes}
Section \ref{sec:software} describes how the CKDMIP software can be
used to perform line-by-line radiative transfer calculations on the
absorption spectra described in section \ref{sec:spectra} to generate
reference fluxes (both spectral and broadband) against which to test
CKD models. Since the code is slow to run,
\code{lw\_fluxes/evaluation1} and \code{sw\_fluxes/evaluation1} of the
FTP contain the broadband upwelling and downwelling fluxes for each of
the 34 scenarios \cite[see Table 2 of][]{Hogan+2020} applied to the 50
profiles of the `Evaluation-1 dataset', in the following files:
%
\begin{verbatim}
ckdmip_evaluation1_lw_fluxes_glacialmax.h5
ckdmip_evaluation1_lw_fluxes_preindustrial.h5
ckdmip_evaluation1_lw_fluxes_present.h5
ckdmip_evaluation1_lw_fluxes_future.h5
ckdmip_evaluation1_lw_fluxes_co2-140.h5
ckdmip_evaluation1_lw_fluxes_co2-280.h5
ckdmip_evaluation1_lw_fluxes_co2-560.h5
ckdmip_evaluation1_lw_fluxes_co2-1120.h5
ckdmip_evaluation1_lw_fluxes_co2-2240.h5
ckdmip_evaluation1_lw_fluxes_ch4-350.h5
ckdmip_evaluation1_lw_fluxes_ch4-700.h5
ckdmip_evaluation1_lw_fluxes_ch4-1200.h5
ckdmip_evaluation1_lw_fluxes_ch4-2600.h5
ckdmip_evaluation1_lw_fluxes_ch4-3500.h5
ckdmip_evaluation1_lw_fluxes_n2o-190.h5
ckdmip_evaluation1_lw_fluxes_n2o-270.h5
ckdmip_evaluation1_lw_fluxes_n2o-405.h5
ckdmip_evaluation1_lw_fluxes_n2o-540.h5
ckdmip_evaluation1_lw_fluxes_cfc11-0.h5
ckdmip_evaluation1_lw_fluxes_cfc11-2000.h5
ckdmip_evaluation1_lw_fluxes_cfc12-0.h5
ckdmip_evaluation1_lw_fluxes_cfc12-550.h5
ckdmip_evaluation1_lw_fluxes_co2-180-ch4-350.h5
ckdmip_evaluation1_lw_fluxes_co2-180-ch4-3500.h5
ckdmip_evaluation1_lw_fluxes_co2-180-n2o-190.h5
ckdmip_evaluation1_lw_fluxes_co2-180-n2o-540.h5
ckdmip_evaluation1_lw_fluxes_co2-2240-ch4-350.h5
ckdmip_evaluation1_lw_fluxes_co2-2240-ch4-3500.h5
ckdmip_evaluation1_lw_fluxes_co2-2240-n2o-190.h5
ckdmip_evaluation1_lw_fluxes_co2-2240-n2o-540.h5
ckdmip_evaluation1_lw_fluxes_ch4-350-n2o-190.h5
ckdmip_evaluation1_lw_fluxes_ch4-350-n2o-540.h5
ckdmip_evaluation1_lw_fluxes_ch4-3500-n2o-190.h5
ckdmip_evaluation1_lw_fluxes_ch4-3500-n2o-540.h5
\end{verbatim}
%
As indicated in the file name, the first four correspond to discrete
scenarios, the next 18 correspond to perturbations of individual gases
while keeping the others at present-day conditions (where the numbers
in the filenames indicate ppmv for CO$_2$, ppbv for CH$_4$ and N$_2$O,
and pptv for CFC11 and CFC12), and the final 12 correspond to
perturbations of pairs of gases together. These files also contain
fluxes (in W~m$^{-2}$) in the 13 `narrow' bands defined in Table 4 of
\cite{Hogan+2020}. In the shortwave, only the first 18 scenarios are
used, and the files also contain calculations at five different solar
zenith angles.

\section{Software}
\label{sec:software}
The latest version of the software package, named
\code{ckdmip-x.y.tar.gz}, may be obtained from the CKDMIP home page,
and contains the \code{ckdmip\_tool}, \code{ckdmip\_lw} and
\code{ckdmip\_sw} programs.  They are coded in Fortran and it have
been tested under Linux. Instructions on compiling are provided in the
\code{README} file within.  The \code{ckdmip\_tool} program performs
various tasks described in section \ref{sec:tool}, such as creating
Rayleigh-scattering and cloud spectral files, and creating solar
spectral irradiance files. The \code{ckdmip\_lw} program can be used
in three ways: to produce reference line-by-line fluxes (section
\ref{sec:lbl}), to merge and store the absorption spectra of
individual gases (section \ref{sec:merge}), or to produce fluxes for
CKD models (section \ref{sec:ckd}).

\subsection{Utility for creating input files}
\label{sec:tool}
The \code{ckdmip\_tool} program performs various tasks according to
the command-line arguments; typically it generates a file at full
spectral resolution from information at coarser resolution. The
arguments available to all tasks are as follows:
%
\begin{description}
\citem{~~--grid input.h5} Read pressure and wavenumber from this file:
output files will use this resolution grid.
\citem{~~-o|--output output.h5} Write the output to the specified file.
\citem{~~--scenario str} Add a \code{scenario} global attribute to the
output file.
\citem{~~--column-range M N} Only process columns \code{M} to \code{N}.
\end{description}
%

To generate the solar spectral irradiance at the spectral resolution
of the gases, provide one of the gas spectra files to the
\code{--grid} argument above, and additionally the following
arguments:
\begin{description}
\citem{~~--ssi TSI ssi.h5} Create a high-resolution solar spectral
irradiance file interpolated from the lower resolution file
\code{ssi.h5}, with a total solar irradiance of \code{TSI} (in
W~m$^{-2}$). Note that CKDMIP uses a nominal value of 1361~W~m$^{-2}$.
\end{description}
A suitable lower-resolution file to use here is
\code{data/mean-ssi\_nrl2.nc} in the CKDMIP software package, which
contains the solar spectral irradiance from \cite{Coddington+2016} in
units of W~m$^{-2}$~cm (i.e.\ irradiance per unit wavenumber) at up to
1-nm resolution.  It is linearly interpolated to the spectral
resolution of the gases. The output spectral irradiance is in units of
W~m$^{-2}$, and sums to the requested \code{TSI} value. The
\code{work/sw/make\_ssi.sh} in the CKDMIP software package
demonstrates the use of this feature.

To generate a file of Rayleigh-scattering optical depth at the same
spectral resolution as gas absorption, use the following argument:
\begin{description}
\citem{~~--rayleigh} Create a Rayleigh scattering spectral file.
\end{description}
The \code{work/sw/make\_rayleigh.sh} demonstrates the use of this
feature.

The following options generate a file containing the high-resolution
spectral properties (layer optical depth, single-scattering albedo and
asymmetry factor) of clouds:
\begin{description}
\citem{~~--cloud WP RE M N input.nc} Generate a cloud spectral file
for a cloud with a water path of \code{WP} (kg m$^{-2}$), and
effective radius of \code{RE} $\mu$m, lying between layers \code{M}
and \code{N} inclusive, reading cloud properties from input file
\code{input.nc}.
%
\citem{~~--delta-cloud WP RE M N input.nc} As above but also apply
delta-Eddington scaling.
%
\citem{~~--absrption-cloud WP RE M N input.nc} As above but for a
non-scattering cloud; the absorption optical depth is saved, while the
single-scattering albedo and asymmetry factor are not.
delta-Eddington scaling.
\end{description}
The \code{data} directory of the package contains suitable input cloud
optical property files: \code{mie\_liquid-droplets\_scattering.nc} for Mie
scattering by liquid droplets, and
\code{baum\_general-habit-mixture\_scattering.nc} for ice particles
\citep{Baum+2014}.


\subsection{Utility for converting input files}
\label{sec:convert}
The \code{ckdmip\_convert} program can convert the files described in
section \ref{sec:spectra}, which contain the optical depth of a
particular gas in each layer, into files containing either the mass
extinction coefficient of the gas, $k$ (the extinction cross section
per unit mass of the gas, in m$^2$~kg$^{-1}$) or the molar extinction
coefficient of the gas, $\epsilon$ (in m$^2$~mol$^{-1}$). These
variables are equivalent to the mass- or molar-\emph{absorption}
coefficient, but the term \emph{extinction} is used in order that the
same variable name could be used in future for Rayleigh scattering.
Usage is as follows:
\begin{verbatim}
ckdmip_convert [--mass|--molar] input.h5 output.h5
\end{verbatim}
where the first argument specifies whether the mass- or
molar-extinction coefficient will be computed, followed by the input
and output file names.  One of the following formulas is applied:
%
\begin{eqnarray}
k_i&=&\frac{gM_d\delta_i}{x_i\left(p_{i+1/2}-p_{i-1/2}\right)};\\
\epsilon_i&=&\frac{gM_d\delta_i}{x_iM_g\left(p_{i+1/2}-p_{i-1/2}\right)},
\end{eqnarray}
%
where index $i$ indicates the values in layer $i$, $\delta_i$ is the
layer optical depth, $g$ is the acceleration due to gravity,
$M_d=0.02897$~kg~mol$^{-1}$ is the molar mass of dry air, $M_g$ is the
molar mass of the gas in question, $x_i$ is the layer-mean mole
fraction of the gas, and $p_{i+1/2}$ is the pressure of the half level
between layers $i$ and $i+1$.  All quantities have SI units.

\subsection{Performing reference longwave flux calculations}
\label{sec:lbl}
For this application, \code{ckdmip\_lw} takes as input one or more
absorption-spectra files (see section \ref{sec:spectra}), optionally
scales them to represent different gas concentrations, and sums the
layerwise optical depths internally to produce the combined optical
depth of all relevant gases. It then performs longwave radiative
transfer calculations and outputs profiles of broadband and,
optionally, spectral fluxes and fluxes in bands. Note that despite the
absorption-spectra files containing temerature, pressure and mole
fraction on both full and half levels, \code{ckdmip\_lw} only reads
pressure and temperature on half levels and mole fraction on full
levels. The radiative transfer calculations involve essentially
applying Eqs.\ 7--12 of \cite{Hogan&2018}. Note that the surface is
assumed to have an emissivity of 1 and the same temperature as the
lowest half level of the atmosphere.

The relevant command-line arguments are as follows:
%
\begin{description}
\citem{~~input.h5} Read in spectral optical depths from the specified
file and add them to the optical depth of the gas mixture. Note that
this can either be a file for a single gas, or the file containing a
merge of several gases (see section \ref{sec:merge}).
%
\citem{~~--scale X input.h5} Read the spectral optical depths from the
specified file, scale them by a factor \code{X}, and add the to the
optical depth of the gas mixture.
%
\citem{~~--conc X input.h5} Read the spectral optical depths and scale
them by a constant such that the nominal surface mole fraction (in
mol~mol$^{-1}$) is \code{X}. This is only possible if the file
contains a \code{reference\_surface\_mole\_fraction} variable
(i.e.\ all gases except H$_2$O and O$_3$).
%
\citem{~~--const X input.h5} Read the spectral optical depths and
scale them in each layer so that the mole fraction is \code{X} in all
layers. This is only possible if the file contains a
\code{reference\_surface\_mole\_fraction} variable.
%
\citem{~~-c|--config config.nam} Read extra configuration information
from the specified namelist file (see below).
%
\citem{~~-o|--output output.h5} Write the output to the specified file.
%
\citem{~~--scenario str} Add a \code{scenario} global attribute to the
output file.
%
\citem{~~--column-range M N} Only process columns \code{M}
to \code{N}.
%
\end{description}
%
Note that multiple input files can be specified. An example of how
\code{ckdmip\_lw} was used in this way to generate the files described
in section \ref{sec:fluxes} may be found in the
\code{work/lw/run\_lw\_lbl\_evaluation.sh} script of the software
package. Since the 50 profiles of the Evaluation-1 dataset are stored
in five files for each gas, each containing 10 profiles, this script
works on 10 profiles at a time, and then uses the NCO utility
\code{ncrcat} to concatenate the profiles into a single file
containing the irradiance profiles for all 50 profiles.

If a namelist file is provided then it can be used to set the variable
names that are read or written, and other aspects of the processing
and output:
%
\begin{verbatim}
&longwave_config
pressure_name = "p_hl",           ! Override default, which is "pressure_hl" 
pressure_scaling = 100.0,         ! Scaling if provided in hPa (default 1)
temperature_name = "t_hl",        ! Override default, which is "temperature_hl"
optical_depth_name = "od",        ! Override default, which is "optical_depth"
planck_name = "planck",           ! Override default, which is "planck_hl"
wavenumber_name = "wn",           ! Override default, which is "wavenumber"
nspectralstride = 2,              ! Skip spectral intervals (default 1)
do_write_spectral_fluxes = false, ! Write spectral fluxes to output (default no)
do_write_planck = false,          ! Write spectral Planck function (default no)
do_write_optical_depth = false,   ! Write merged optical depth (default no)
band_wavenumber1(1:13) = 0, 350, 500, 630, 700, 820, 980, 1080, 1180, &
  1390, 1480, 1800, 2080,         ! Specify lower wavenumbers of output bands
band_wavenumber2(1:13) = 350, 500, 630, 700, 820, 980, 1080, 1180, 1390, &
  1480, 1800, 2080, 3260,         ! Specify upper wavenumbers of output bands
nangle = 4                        ! Number of angles per hemisphere
input_planck_per_sterad = true,   ! Default false (Planck function in W m-2)
iverbose = 3,                     ! Set verbosity level from 1 to 5
/
\end{verbatim}
Here, the \code{band\_wavenumber1} and \code{band\_wavenumber2}
entries specify that in addition to broadband and (optionally) full
spectral output, the irradiance profiles should also be provided in
bands bounded by the wavenumbers in these two vectors (in
cm$^{-2}$). This is useful to evaluate CKD schemes in individual
bands.

The \code{nangle} entry specifies the number of angles to use per
hemisphere in the longwave radiative transfer, with the angles and
weights selected according to Gauss-Legendre quadrature.  The maximum
number of angles is 8.  Commonly one refers to an \emph{N-stream
  scheme} where \emph{N} is twice \code{nangle}.  The default value of
\code{nangle=0} means to use one angle per hemisphere, but using a
zenith angle of $\pm$53$^\circ$.  This gives better agreement with the
higher order approximations than \code{nangle=1}, which uses a zenith
angle of $\pm$60$^\circ$.

\subsection{Performing reference shortwave flux calculations}
\label{sec:lbl_sw}
The executable \code{ckdmip\_sw} works in a similar fashion to
\code{ckdmip\_lw} but performs shortwave radiation calculations. The
arguments are essentially the same except that one of the input files
must be a Rayleigh scattering file produced by \code{ckdmip\_tool} as
described in section \ref{sec:tool}, and the following additional
argument must be provided:
\begin{description}
\citem{~~--ssi ckdmip\_ssi.h5} Specify the location of a spectral
solar irradiance file produced using \code{ckdmip\_tool} at the same
spectral resolution as the gas absorption files.
\end{description}
An example may be found in the
\code{work/sw/run\_sw\_lbl\_evaluation.sh} script. The namelist
options are similar to the longwave but with the following additions:
\begin{verbatim}
&shortwave_config
single_scattering_albedo_name = "ssa_sw",! Override default "single_scattering_albedo"
incoming_flux_name = "incoming_sw",      ! Override default "incoming_flux"
rayleigh_optical_depth_name = "r_od",    ! Override default "rayleigh_optical_depth"
surf_albedo = 0.15,                      ! Surface albedo to use
use_mu0_dimension = true,                ! Use mu0 dimension in output file
cos_solar_zenith_angle(1:5) = 0.1,0.3,0.5,0.7,0.9,   ! Use these mu0 values
do_write_spectral_boundary_fluxes = true,! Write also spectral TOA-up & surface-down
do_write_direct_only = false,            ! Write only direct downward flux
i_spectral_level_index = 10,11,12,       ! Write spectra only at these levels
/
\end{verbatim}
The final option specifies what values are used for the cosine of the
solar zenith angle, $\mu_0$, and the option before that specifies
whether \code{mu0} is an additional dimension in the output file (by
default it is not).

\subsection{Merging gases}
\label{sec:merge}
This usage of \code{ckdmip\_lw} allows the optical depth of
`hybrid' or `composite' gases to be created, by combining the
following option with those in section \ref{sec:lbl}.
%
\begin{description}
\citem{~~-m|--merge-only} Store the merged optical depth of the gas
mixture in the output file, but do not perform any radiative transfer
calculations.
\end{description}
%
See the scripts \code{work/lw/test\_merge\_only.sh} and
\code{test\_lw\_lbl\_merged.sh} in the CKDMIP software package.

\subsection{Producing fluxes for CKD models}
\label{sec:ckd}
The \code{ckdmip\_lw} and \code{ckdmip\_sw} programs can also work on
files produced by CKD models, by replacing all the options in section
\ref{sec:lbl} related to reading in absorption spectra with a single
option:
%
\begin{description}
\citem{~~--ckd input.nc} Read the optical depth and other necessary
variables in each g-point of a CKD model from the specified file.
\end{description}
%
In addition to broadband fluxes, the output file will include the flux
profiles for each of the g-points used by the CKD model.

The idea is that the CKD model produces a description of its
approximation of the optical properties of the atmosphere, but does
not perform any radiative transfer.  By using \code{ckdmip\_lw} and
\code{ckdmip\_sw} to do the radiative transfer, we ensure that the
differences with line-by-line fluxes and heating rates (produced by
the same programs in sections \ref{sec:lbl} and \ref{sec:lbl_sw}) are
due only to the treatment of gas absorption. The format of the input
file is described in section \ref{sec:requirements}.

Once radiative transfer has been run on the output from a CKD model,
the fluxes and heating rates may be compared to those of the
line-by-line reference calculations (section \ref{sec:fluxes}).  The
Matlab code in the \code{matlab} directory of the package illustrates
how this may be done.

\section{Requirements of participants}
\label{sec:requirements}
The purpose of CKDMIP is primarily to compare \emph{CKD tools,} which
we define as a method (which may be fully automated or involve some
hand-tuning) for generating individual CKD models, with some means to
control the trade-off between accuracy and the number of g-points. An
individual \emph{CKD model} is a software component for converting
profiles of temperature, pressure and gas concentrations into optical
depth profiles in $N$ g-points (i.e.\ to produce the input illustrated
in section \ref{sec:ckd}), where $N$ is specific to a particular CKD
model.

Some CKDMIP participants will have a CKD tool that, with relative
ease, can generate CKD models targeted at a particular application
and adjust the trade-off between accuracy and efficiency. Section 4.2
of \cite{Hogan+2020} describes the CKD models that would ideally be
generated by such participants.  Other participants will have tools
that are less flexible, perhaps involving hand-tuning, so would only
be able to generate a small number of models.

Either way, for each generated CKD model, the scenarios that it should
be run on are given in section 4.2 of \cite{Hogan+2020}.  Thus for
each model--scenario pair, the participant should submit a file
containing the optical depth profiles in each g-point that can be used
to compute fluxes as described in section \ref{sec:fluxes}.
%
These files should have names consisting of seven fields separated
by underscores:
%
\begin{verbatim}
TOOL_DATASET_BAND_APPLICATION_CONFIGURATION_optical-depth_SCENARIO.nc
\end{verbatim}
%
where the fields are:
\begin{description}
\citem{~~TOOL} A string identifying the CKD tool and possibly its
version number, e.g. \code{ecrad-rrtmg} for the ecRad implementation
of the RRTMG gas optics scheme, or \code{ecckd-v0.1} for the ECMWF
tool for generating CKD models (when it's available).
\citem{~~DATASET} Either \code{evaluation1} or \code{evaluation2}
depending on which set of 50 profiles are being processed.
\citem{~~BAND} Either \code{lw} or \code{sw}.
\citem{~~APPLICATION} The `application' from the three listed in Table
1 of \cite{Hogan+2020}: \code{limited-area-nwp}, \code{global-nwp} or
\code{climate}.
\citem{~~CONFIGURATION} Participants are requested to generate several
CKD models to explore the accuracy--efficiency trade-off: this string
is used to distinguish between these different configurations, and
would typically include the total number of g-points.  For example,
RRTMG could use \code{narrow-140} since its bands fall into the
`narrow bands' shown in Table 4 of \cite{Hogan+2020} and it uses a
total of 140 g-points, while another scheme could use \code{wide-80}.
\citem{~~SCENARIO} The gas-concentration scenario to which the CKD
model has been applied \cite[see Table 2 and section 4.2
  of][]{Hogan+2020}, using strings shown in the last field of the file
name shown in section \ref{sec:fluxes}.
\end{description}
%
The file must be in NetCDF format (either `classic' or HDF5).  Its
format is best illustrated by showing the \code{ncdump} header output
on conforming files produced by ecRad \cite[]{Hogan&2018}, which uses
the RRTMG CKD model of \cite{Mlawer+1997} for its gas optics. In the
case of the `Evaluation-1' dataset in present-day conditions we have
the following in the longwave:
%
\begin{verbatim}
netcdf ecrad-rrtmg_evaluation1_lw_climate_narrow-140_optical-depth_present {
dimensions:
    column = UNLIMITED ; // (50 currently)
    level = 54 ;
    gpoint_lw = 140 ;
    half_level = 55 ;
variables:
    float pressure_hl(column, half_level) ;
        pressure_hl:long_name = "Pressure on half-levels" ;
        pressure_hl:units = "Pa" ;
    float od_lw(column, level, gpoint_lw) ;
        od_lw:long_name = "Clear-sky longwave optical depth" ;
        od_lw:units = "1" ;
    float planck_hl(column, half_level, gpoint_lw) ;
        planck_hl:long_name = "Planck function on half-levels" ;
        planck_hl:units = "W m-2" ;

// global attributes:
        :title = "Spectral radiative properties from the ecRad radiation model" ;
        :source = "ecRad offline radiation model" ;
...
\end{verbatim}
%
The layers must start at the top-of-atmosphere, but otherwise the
format is quite flexible.  The dimension names are not important;
\code{ckdmip\_lw} only cares about the ordering of dimensions of the
three variables. The column-like dimension is shown here as unlimited
but this is not a requirement. In this example, the number of longwave
g-points is 140, but this number will of course be different for other
CKD models. The global attributes are not read.

The three variables shown here provide the information needed to
perform radiative transfer by \code{ckdmip\_lw} in section
\ref{sec:fluxes}. They are the pressure and the Planck function at
half levels (layer interfaces) and the optical depth at full levels
(layers).  The Planck function for each g-point is the integral across
all parts of the spectrum that contribute to that g-point.  It should
be in units of W~m$^{-2}$, i.e.\ the energy emitted by a black body
through a horizontal plane in a particular wavelength interval.  If
Planck function is instead provided as the energy emitted by a black
body per unit solid angle (i.e.\ W~m$^{-2}$~sr$^{-1}$) then use
\code{input\_planck\_per\_sterad=true} in the namelist described in
section \ref{sec:lbl}; this simply multiplies the input by a factor of
$\pi$ before using it. The actual names of these three variables can
be anything, since the namelist file
% (see section \ref{sec:lbl})
specifies the names of the variables to look for.

The following is an example file for the shortwave
%
\begin{verbatim}
netcdf ecrad-rrtmg_evaluation1_sw_climate_narrow-112_optical-depth_present {
dimensions:
    column = UNLIMITED ; // (50 currently)
    level = 54 ;
    half_level = 55 ;
    gpoint_sw = 112 ;
variables:
    float pressure_hl(column, half_level) ;
        pressure_hl:long_name = "Pressure on half-levels" ;
        pressure_hl:units = "Pa" ;
    float incoming_sw(column, gpoint_sw) ;
        incoming_sw:long_name = "Incoming shortwave flux at top-of-atmosphere...
        incoming_sw:units = "W m-2" ;
    float od_sw(column, level, gpoint_sw) ;
        od_sw:long_name = "Clear-sky shortwave optical depth" ;
        od_sw:units = "1" ;
    float ssa_sw(column, level, gpoint_sw) ;
        ssa_sw:long_name = "Clear-sky shortwave single scattering albedo" ;
        ssa_sw:units = "1" ;

// global attributes:
        :title = "Spectral radiative properties from the ecRad radiation model" ;
        :source = "ecRad offline radiation model" ;
\end{verbatim}
%
Rather than the Planck function being provided, the
\code{incoming\_sw} variable provides the solar irradiance at
top-of-atmosphere in each g-point.  These values ought to sum along
the \code{gpoint\_sw} dimension to the nominal total solar radiance
used in CKDMIP of 1361~W~m$^{-2}$.  Rayleigh scattering may be
specified in two ways: in the example above, the optical depth
includes both gas absorption and Rayleigh scattering, and the Rayleigh
fraction is provided by the single scattering albedo
variable. Alternatively, the Rayleigh optical depth may be provided as
an additional variable whose name is given by the
\code{rayleigh\_optical\_depth\_name} namelist variable, as shown in
section \ref{sec:lbl_sw}. In this case, the main optical depth
variable is taken to mean only the optical depth due to gas
absorption.

One additional file is required for each CKD model, describing what
fraction of different parts of the spectrum contribute to each
g-point. This will be used as part of the investigation of the effect
of clouds described in section 4.4 of \cite{Hogan+2020}.  In the
longwave this should be expressed in 326 spectral intervals at a
resolution of 10~cm$^{-1}$ between 0 and 3260~cm$^{-1}$, and in the
shortwave it should be provided in 995 spectral intervals at a
resolution of 50~cm$^{-1}$ between 250 and 50,000~cm$^{-1}$. This is
commensurate with the spectral scale at which the optical properties
of clouds varies, and also ensures that each spectral interval lies
entirely within one of the `narrow' spectral bands defined in Tables 4
and 5 of \cite{Hogan+2020}. The file name should be of the form
%
\begin{verbatim}
TOOL_BAND_APPLICATION_CONFIGURATION_spectral-definition.nc
\end{verbatim}
%
and an example file for the ecRad implementation of RRTMG is available
from the CKDMIP web site, with a \code{ncdump} header output of:
%
\begin{verbatim}
netcdf ecrad-rrtmg_lw_climate_narrow-140_spectral-definition {
dimensions:
    gpoint_lw = 140 ;
    wavenumber = 326 ;
variables:
    float wavenumber1(wavenumber) ;
        wavenumber1:long_name = "Lower wavenumber bound of spectral interval" ;
        wavenumber1:units = "cm-1" ;
    float wavenumber2(wavenumber) ;
        wavenumber2:long_name = "Upper wavenumber bound of spectral interval" ;
        wavenumber2:units = "cm-1" ;
    float gpoint_fraction(gpoint_lw, wavenumber) ;
        gpoint_fraction:long_name = "Fraction of spectrum contributing to each g-point" ;

\end{verbatim}
%
The variables \code{wavenumber1} and \code{wavenumber2} define the
bounds of the high-resolution spectral intervals.  The variable
\code{gpoint\_fraction} the defines what fraction of the spectrum at
this high spectral resolution contributes to each g-point. This
variable should sum to unity along the \code{wavenumber}
dimension. Note that in the longwave, the fractions should \emph{not}
be weighted by the Planck function at a particular temperature; since
the spectral intervals are at high resolution, they can be weighted by
the Planck function afterwards, using any temperature desired.
Likewise, in the shortwave they should not be weighted by the solar
spectrum.

\begin{thebibliography}{00}
\markright{References}
%
\harvarditem{Baum et~al.}{2014}{Baum+2014}Baum, B. A., P. Yang,
A. J. Heymsfield, A. Bansemer, A. Merrelli, C. Schmitt, and C. Wang,
2014: Ice cloud bulk single-scattering property models with the full
phase matrix at wavelengths from 0.2 to 100
$\mu$m. \textit{J. Quant.\ Spectrosc.\ Radiat.\ Transfer,}
\textbf{146,} 123--139.
%
\harvarditem{Clough et~al.}{2005}{Clough+2005}Clough, S. A.,
M. W. Shephard, E. J. Mlawer, J. S. Delamere, M. J. Iacono,
K. Cady-Pereira, S. Boukabara and P. D. Brown, 2005: Atmospheric
radiative transfer modeling: a summary of the AER
codes. \textit{J. Quant.\ Spectrosc.\ Radiat.\ Transfer,} \textbf{91,}
233--244.
%
\harvarditem{Coddington et~al.}{2016}{Coddington+2016}Coddington, O.,
  J. L. Lean, P. Pilewskie, M. Snow and D. Lindholm, 2016: A solar
  irradiance climate data
  record. \textit{Bull.\ Am.\ Meteorol.\ Soc.,} \textit{97,}
  1265--1282.
%
\harvarditem{Hogan and Bozzo}{2018}{Hogan&2018}Hogan, R. J., and Bozzo,
  A., 2018: A flexible and efficient radiation scheme for the ECMWF
  model. \textit{J. Adv. Modeling Earth Sys.,} \textbf{10,}
  1990--2008.
%
\harvarditem{Hogan and Matricardi}{2020}{Hogan+2020}Hogan, R. J., and
M. Matricardi, 2020: Evaluating and improving the treatment of gases
in radiation schemes: the Correlated K-Distribution Model
Intercomparison Project (CKDMIP). \textit{Geosci.\ Model Dev.,}
\textbf{13,} 6501--6521.
%
\harvarditem{Mlawer et~al.}{1997}{Mlawer+1997} Mlawer, E. J.,
S. J. Taubman, P. D. Brown, M. J. Iacono, and S. A. Clough, 1997:
Radiative transfer for inhomogeneous atmospheres: RRTM, a validated
correlated-k model for the longwave.  \textit{J. Geophys.\ Res.\ Atmos.,}
\textbf{102,} 16\,663--16\,682.
%
\end{thebibliography}
\end{document}
